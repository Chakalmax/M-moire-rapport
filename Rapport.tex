\documentclass[letterpaper, 12pt]{report}
\usepackage[top = 1.8cm, left = 3cm, right = 3cm ]{geometry}
\usepackage[pdftex]{graphicx}
\usepackage{amsmath}
\usepackage{amsthm}
\usepackage{url}
\usepackage{tikz}
\usepackage{float}
\usepackage{listings}
\usepackage[T1]{fontenc}
\usepackage[utf8]{inputenc}
\usepackage{epigraph}
\usepackage{fancyhdr}
\usepackage{gensymb}
\usepackage{rotating}
\usepackage[french,ruled,vlined]{algorithm2e}
\usepackage{longtable}
\usepackage[english,frenchb]{babel}
\usepackage{newfloat}
\usepackage{enumitem}
\usepackage{amsfonts}

\DeclareFloatingEnvironment[placement={!ht},name=Liste]{mylist}
\newtheorem{mydef}{Definition}

\def\changemargin#1#2{\list{}{\rightmargin#2\leftmargin#1}\item[]}
\let\endchangemargin=\endlist 

\newcommand{\alinea}{
\hspace*{0.5cm}}

\renewcommand*\sfdefault{phv}
\renewcommand*\rmdefault{ppl}

\renewcommand\epigraphflush{flushright}
\renewcommand\epigraphsize{\normalsize}
\setlength\epigraphwidth{0.7\textwidth}

\definecolor{titlepagecolor}{RGB}{255,20,20}

\DeclareFixedFont{\titlefont}{T1}{phv}{\seriesdefault}{n}{0.375in}
  

\makeatletter
\@addtoreset{section}{part}

\renewcommand{\partname}{Partie}


% The following code is borrowed from: http://tex.stackexchange.com/a/86310/10898

\newcommand\titlepagedecoration{%
\begin{tikzpicture}[remember picture,overlay,shorten >= -10pt]

\coordinate (aux1) at ([yshift=-50pt]current page.north east);
\coordinate (aux2) at ([yshift=-380pt]current page.north east);
\coordinate (aux3) at ([xshift=-5cm]current page.north east);
\coordinate (aux4) at ([yshift=-130pt]current page.north east);
\coordinate (aux5) at ([yshift=-4cm]current page.north west);
\coordinate (aux6) at ([xshift=4cm]current page.north west);


\begin{scope}[titlepagecolor!40,line width=12pt,rounded corners=12pt]
\draw
  (aux1) -- coordinate (a)
  ++(225:5) --
  ++(-45:5.1) coordinate (b);
\draw[shorten <= -10pt]
  (aux3) --
  (a) --
  (aux1);
\draw[opacity=0.6,titlepagecolor,shorten <= -10pt]
  (b) --
  ++(225:2.2) --
  ++(-45:2.2);
\draw[opacity=0.5,titlepagecolor,shorten <= -15pt]
  (aux5) --
  (aux6);
\end{scope}
\draw[titlepagecolor,line width=8pt,rounded corners=8pt,shorten <= -10pt]
  (aux4) --
  ++(225:0.8) --
  ++(-45:0.8);

\begin{scope}[titlepagecolor!70,line width=6pt,rounded corners=8pt]
\end{scope}
\end{tikzpicture}%
}

\begin{document}
\begin{titlepage}

\noindent


\newgeometry{bottom = 2cm, top = 2.5cm}
\begin{center}
\includegraphics[scale=0.2]{umonslogo}\\
\vspace*{0.7cm}
\includegraphics[scale=0.32]{fs-logo}\\
\vspace*{2.5cm}
\titlefont Mémoire\\~\\{\LARGE  Data Repairing\\}~\\~\\{\large} \par
\end{center}
\vspace*{3.5cm}
\hfill
\begin{minipage}{0.18\linewidth}
  \begin{flushright}
   \rule{0.5pt}{75pt}
  \end{flushright}
\end{minipage}
\begin{minipage}{0.8\linewidth}
\begin{flushleft}
\textsf{\textbf{Projet réalisé par:}} Maxime Van Herzeele\\
\textsf{\textbf{Année académique:}} 2017-2018\\
\textsf{\textbf{Directeur de Mémoire:}} Jeff Wijsen\\
%\textsf{\textbf{Rapporteurs}} Pierre Hauweele \& Tom Mens\\
\textsf{\textbf{Section:}} 2$^{ème}$ Bloc Master en Sciences Informatiques
\end{flushleft}
\end{minipage}
\vspace*{\fill}
\begin{center}
Faculté des Sciences $\bullet$ Université de Mons $\bullet$ Place du Parc 20 $\bullet$ B-7000 Mons
\end{center}
\titlepagedecoration
\end{titlepage}

\newgeometry{top = 3cm, left = 2.5cm, right = 2.5cm}

\pagestyle{fancy}
\lhead{Maxime Van Herzeele}
\rhead{MAB2 Sciences Informatiques}
\cfoot{\thepage}

\pagenumbering{roman} \setcounter{page}{1} 
\section*{Remerciements} 
\vspace*{0.8cm}
\addcontentsline{toc}{section}{Remerciements} 
Todo : remerciement
\newpage

\tableofcontents
\pagebreak

\chapter{Introduction}

\pagenumbering{arabic} \setcounter{page}{1} 

De nombreuse entreprises et institutions récoltent, conservent et utilisent un nombre importants de données. Ces données sont parfois considérées comme inexactes car elles ne respectent pas certaines règles appelées \emph{contraintes d'intégrité}. Parfois ce sont ces contraintes d'intégrités qui sont incorrectes. Ces deux types d'erreurs sont contraignantes à l'utilisation des données.

Le \emph{data repairing} consiste à identifier les données et les contraintes d'intégrités et à les corriger. Dans ce mémoire, nous allons analyser une technique de data repairing appelés \emph{Modèle de réparation $\theta$-tolérant}. Cette technique est tiré d'un article scientifique Dans un premier temps, nous allons rappelé différentes notions importantes du modèle relationnel tel que les contraintes d'intégrités. Ensuite nous allons présenter des bases de données afin d'illustrer les notions rappelés précédemment. + TODO(continué l'intro au fur et à mesure des chapitres)

\chapter{?}

\section{Le modèle relationnel}


\alinea Il est impératif que les données soient organisées et que les liens entre les informations soient modélisés. Cela se fait par le modèle relationnel avec contraintes d'intégrités. C'est l'un des modèles les plus utilisés. Le but de ce mémoire n'est pas d'expliquer ce modèle mais nous allons rappeler quelques notions que nous allons régulièrement utilisés.

Dans le modèle relationnel, nous avons \cite{Integrity}
\begin{itemize}
\item Un alphabet \emph{A} de symboles de prédicats. Chaque symbole de l'alphabet est unique
\item Un set de \emph{contraintes d'intégrités}  $\varphi$ qui sont des règles définissant la cohérence d'une donnée ou d'un ensemble de données de la BD. Ce sont des assertions des prédicats de l'alphabet.
\end{itemize}

\begin{mydef}
Soit un schéma de relation $R$ avec comme attributs $attr(R)$ Soit un ensemble $\mathbb{P}$ de prédicat $P$ de la forme $v_1 \phi V_2$ ou $v1 \phi c$ avec $v_1,v_2 \in t_x.A$ , $x \in \{\alpha,\beta\}$ , $t_\alpha,t_\beta \in R$, $A \in attr(R)$, $c$ est une constante et $\phi \in \{ =,<,>,\leq, \geq, \neq \}$ est un opérateur.  Une \textbf{contrainte de déni(denial constraint)}\cite{main}:

TODO
\end{mydef}
%
%\subsection{Les contraintes d'intégrité}
%\begin{mydef}
%Une \textbf{contrainte d'intégrité} $\varphi$ est une règle définissant la cohérence d'une donnée ou d'un ensemble de données de la BD
%\end{mydef}

\section{Les bases de données}
Dans cette section nous présenterons les bases de données que nous utiliserons comme exemple afin d'illustrer les différentes notions que nous aborderons.
\subsection{Exemples de bases de données}

La première base de données que nous allons utilisés vient de la source principale de ce mémoire \cite{main}.

\begin{table}[H]
	\centering
	\begin{tabular}{|c|c c c c c c|}
	\hline
	    & Name & BirthDay & Cellphone Number & Year & Income & Tax\\
	\hline
	 t1 & Ayres & 8-8-1984 & 322-573 & 2007 & 21k & 0\\
	 t2 & Ayres & 5-1-1960 & ***-389 & 2007 & 22k & 0 \\
	 t3 & Ayres & 5-1-1960 & 564-389 & 2007 & 22k & 0 \\
	 t4 & Stanley & 13-8-1987 & 868-701 & 2007 & 23k & 3k\\
	 t5 & Stanley & 31-7-1983 & ***-198 & 2007 & 24k & 0\\
	 t6 & Stanley & 31-7-1983 & 930-198 & 2008 & 24k & 0\\
	 t7 & Dustin & 2-12-1985 & 179-924 & 2008 & 25k & 0 \\
	 t8 & Dustin & 5-9-1980 & ***-870 & 2008 & 100k & 21k \\
	 t9 & Dustin & 5-9-1980 & 824-870 & 2009 & 100k & 21k \\
	 t10 & Dustin & 9-4-1984 & 387-215 & 2009 & 150k & 40k \\
	 \hline
	\end{tabular}
	\caption{\label{tableMain} Table de l'article de référence}.
\end{table}

La seconde base de données que nous allons utilisés est inventée de toute pièce. C'est un exemple de base de données que l'on peut retrouver dans un service public.
Les attributs de la tables sont:
\begin{itemize}
\item \textbf{Niss:} Le numéro national de la personne.
\item \textbf{Nom:} Le nom de la personne.
\item \textbf{Prénom:} Le prénom de la personne.
\item \textbf{Date\_naissance:} La date à laquelle la personne est née.
\item \textbf{Date\_décès:} La date à laquelle la personne est décédée.
\item \textbf{statut\_civil:} Le statut civil de la personne (exemple: célibataire, marié, décédé, divorcé,...).
\end{itemize}

\begin{table}[H]
	\centering
	\begin{tabular}{|c|c c c c c c|}
	\hline
	    & Niss & Nom & Prénom & Date\_naissance & Date\_décès & statut\_civil\\
	\hline
	 t1 & 14050250825 & Dupont & Jean & 14-05-1902 & 18-05-1962 & décédé \\
	\hline
	 
	 \hline
	\end{tabular}
	\caption{\label{tableMain} Table Person}.
\end{table}

\section{}

\chapter{Data Repairing}
\section{Variations de contraintes d'intégrités}
Insertion et délétion.
\section{Modèle $\theta$-tolerant}
\section{Autres modèles}
\subsection{Holistic}
 et d'autres

\chapter{Implémentation \& comparaison de modèles}
\chapter{Conclusion}

\bibliographystyle{plain}


\bibliography{biblio}

\newpage
\appendix
\end{document}